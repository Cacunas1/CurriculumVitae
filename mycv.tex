% Created 2018-01-23 Tue 21:55
% Intended LaTeX compiler: pdflatex
\documentclass[10pt,letterpaper,sans]{moderncv}
\moderncvstyle{casual}
\moderncvcolor{blue}
\usepackage[utf8]{inputenc}
\usepackage[scale=0.75]{geometry}
\setlength{\hintscolumnwidth}{4cm}
\name{Cristian Andrés}{Acuña Sepúlveda}
\title{Curriculum Vitae}
\address{El Nogal 2659}{Viña del Mar, Región de Valparaíso}{Chile}
\phone[mobile]{+56~9~6403~7058}
\email{c.acuna.sepulveda@gmail.com}
\photo[64pt][0.4pt]{picture.jpg}
\quote{El hombre está condenado a ser libre; porque una vez arrojado al mundo,
él es responsable de todo lo que hace - Jean Paul Sartre}
\begin{document}
\makecvtitle

\section{Información Personal}
\label{sec:orgfa4e788}
\begin{itemize}
\item RUT: 17.922.906-4
\item Nacionalidad: Chileno
\item Fecha de Nacimiento: 22 de Abril de 1991
\item Estado Civil: Casado
\item Dirección: El Nogal 2659, Block 41 Departamento 23, Viña del Mar
\item email: c.acuna.sepulveda@gmail.com
\item Teléfono: +56 9 6403 7058
\end{itemize}
\section{Presentación}
\label{sec:org93c8601}
  Soy egresado de Ingeniería Ejecución en Electrónica de la UTFSM. Me interesan
las áreas de Visión por Computador, Aprendizaje de Máquina y Control Digital.
Me gusta trabajar en equipo, creo que es donde más se aprende. Tengo experiencia
desarrollando sistemas desde la cotización de equipos, concertando reuniones con
proveedores, hasta las pruebas de campo. Estoy ansioso de pertenecer a vuestra
fuerza de trabajo, y aprender de su cultura empresarial y mis posibles colegas. 
\section{Educación}
\label{sec:org0bb4669}
\cventry{2017}{Egresado de Ingeniería Ejecución en Electrónica}{UTFSM}{
  Valparaíso}{}{}
\cventry{2017}{Licenciado en Ciencias de la Ingeniería Electrónica}{
  UTFSM}{Valparaíso}{}{}
\section{Experiencia Laboral}
\label{sec:org6f89432}
\cventry{Mayo, 2017 - Julio, 2017}
{Desarrollador Móvil - Del Mar TG}
{Del Mar TG}
{Viña del Mar (Remoto)}
{Modalidad - Practicante}
{Mantención de aplicaciones para iOS en Objective-C y Swift}

\cventry{Enero, 2015 - Julio, 2015}
{Ingeniero de Desarrollo}
{Lemsystem SpA}
{Viña del Mar}
{Modalidad - Part-time}
{
\begin{itemize}
\item Desarrollo de Sistema de Monitoreo remoto inalámbrico para medición de caudal de
canales de regadío, usando un sistema de energía solar para su autonomía. Evaluación
del sensor, y programación del Firmware.
\item Desarrollo de Sistema de Monitoreo remoto inalámbrico para medición de nivel de agua en pozos de 
regadío, usando un sistema de energía solar para su autonomía. Evaluación
del sensor, y programación del Firmware.
\end{itemize}
}
\cventry{Marzo, 2013 - Octubre, 2014}
{Programador}
{Centro Científico Tecnológico de Valparaíso, UTFSM}
{Valparaíso}
{Modalidad - Part-time}
{Programación de aplicaciones de simulación de sistemas físicos de alta energía y análisis estadístico 
usando frameworks \emph{Geant4} y \emph{ROOT} respectivamente.}

\cventry{2011 (Primer y Segundo Semestre)}
{Profesor Ayudante}
{UTFSM}
{Valparaíso - Viña del Mar - Quilpué}
{Modalidad - Part-time}
{
\begin{itemize}
\item Ayudante Matemáticas IV (MAT024)
\begin{itemize}
\item Cálculo Vectorial
\item Integrales Múltiples
\item Ecuaciones Diferenciales Parciales
\end{itemize}
\item Ayudante Matemáticas I (MAT021)
\begin{itemize}
\item Límites
\item Derivadas Ordinarias, Implícitas
\item Funciones e Identidades Trigonométricas
\item Números Complejos
\end{itemize}
\item Profesor Particular
\begin{itemize}
\item Clases de Física, Matemáticas, Programación y Electrónica Básica
a alumnas de Enseñanza Media, Centros de Formación Técnica (AIEP),
Institutos Profesionales (Duoc, Inacap) y Universidades (UTFSM, PUCV).
\end{itemize}
\end{itemize}}
\section{Idiomas}
\label{sec:org669ecce}
\cvitemwithcomment{Inglés}{Avanzado}{Lectura, Escritura y Traducción Nivel Alto,
  Conversacional Nivel Medio}
% \begin{itemize}
% \item Inglés 
% \begin{itemize}
% \item Escrito: Avanzado
% \item Lectura: Avanzado
% \item Traducción: Medio
% \item Hablado: Medio
% \end{itemize}
% \end{itemize}
\section{Competencias Computacionales}
\label{sec:org544fd0f}
\begin{itemize}
\item \textbf{C/C++}
\begin{itemize}
\item OpenCV
\end{itemize}
\item \textbf{Java}
\item \textbf{Python}
\begin{itemize}
\item Scipy, Numpy
\end{itemize}
\item \textbf{MATLAB}
\item \textbf{\LaTeX}
\item \textbf{iOS}
\begin{itemize}
\item Objective-C
\item Swift
\end{itemize}
\item \textbf{Microsoft Office}
\begin{itemize}
\item Word
\item PowerPoint
\item Excel
\begin{itemize}
\item Nivel Intermedio (uso básico de Macros, análisis de sensibilidad de simplex y otros
algoritmos de optimización, cálculo de VAN, TIR)
\end{itemize}
\end{itemize}
\end{itemize}
\section{Enlaces}
\begin{itemize}
\item \url{https://github.com/cacunas}
\item \url{https://www.linkedin.com/in/cacunasepulveda/} 
\item \url{https://twitter.com/cacunas_}
\end{itemize}
\end{document}